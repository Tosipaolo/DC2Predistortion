\documentclass{article}

\usepackage[english]{babel}


\usepackage[letterpaper,top=2cm,bottom=2cm,left=3cm,right=3cm,marginparwidth=1.75cm]{geometry}

% Useful packages
\usepackage{amsmath}
\usepackage{graphicx}
\usepackage{setspace}
\usepackage{ragged2e}
\usepackage[colorlinks=true, allcolors=blue]{hyperref}

\begin{document}

\begin{titlepage}
\newcommand{\HRule}{\rule{\linewidth}{0.3mm}}
\center
\centering
\Large {\uppercase {\textbf{DIGITAL COMMUNICATION II}}}\\[0.5cm]
\vspace{+25mm}
\textbf{\text{ADAPTIVE PREDISTORTION}}\\
\vspace{15mm}
\Large{\text{Course held by}\\
\vspace{5mm}
\Large{\textbf{Prof. Arnaldo Spalvieri}}}\\
\Large{\textbf{Supervisor: Davide Scazzoli}}
\vspace{10mm}
\textbf{}\\
\emph{By}\\
\vspace{5mm}
\Large \textbf{Paolo - }\\
\Large \textbf{Riccardo - }\\
\Large \textbf{Mayura Muruga Pichai - 10826970}\\
\textbf{}\\
\vspace{2mm}
\text{}\\
\text{}\\
\textbf{}\\
\includegraphics[width=0.5\textwidth]{Figures/logo.png}\\
\vspace{4mm}
\text{Academic Year 2021/2022}

\end{titlepage}
\pagebreak

\begin{spacing}{1.5}
\section{INTRODUCTION}
High power amplification of linear modulation schemes which exhibit fluctuating envelopes, invariably leads to the generation of distortion and intermodulation products. In order to avoid these effects, maintaining both power and spectral efficiency, it is necessary to use linearization techniques. By using linearization techniques, the amplifier can be operated near the saturation with good efficiency and linearity.

The power amplifier is modelled as a memoryless non linear system, hence itsdistortion can be compensated by putting 
 before the amplifier a pre-distorter that, in this case,
is a memoryless non linear system that inverts the non liner transformation made by the amplifier.

\section{CREST FACTOR}






\end{spacing}
\end{document}
